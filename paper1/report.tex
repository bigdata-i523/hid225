\documentclass[sigconf]{acmart}

\usepackage{hyperref}

\usepackage{endfloat}
\renewcommand{\efloatseparator}{\mbox{}} % no new page between figures

\usepackage{booktabs} % For formal tables

\settopmatter{printacmref=false} % Removes citation information below abstract
\renewcommand\footnotetextcopyrightpermission[1]{} % removes footnote with conference information in first column
\pagestyle{plain} % removes running headers

\begin{document}
\title{Optimizing Mass Transit Bus Routes with Big Data}


\author{Matthew Schwartzer}
\affiliation{%
  \institution{Indiana University, School of Informatics, Computing, \& Engineering}
  \streetaddress{919 E 10th St}
  \city{Bloomington} 
  \state{Indiana} 
  \postcode{47408}
}
\email{mabschwa@indiana.edu}


\begin{abstract}
Optimized public bus systems can reduce congestion and greenhouse gas emissions, while offering a safe, affordable, and convenient way to travel; however, in many cities people prefer to take private transportation over public buses. Big data analytics is important because new data collection and analytical techniques can efficiently optimize service routes, schedules, and infrastructure. From a customer's perspective, public buses are only an option if they are fast, reliable, and comfortable. Fortunately, big data analytical methods such as human mobility mining \& clustering, ant colony and genetic algorithms, and Monte Carlo simulations make it more possible than ever to offer a dynamic and convenient public bus choice to compete with private vehicle transportation. 

\end{abstract}

\keywords{i523, hid225, \LaTeX, big data, bus route optimization}


\maketitle

\section{Introduction}

Optimized public bus transit is one of the keys to bring a better quality of life (QOL) to small and medium sized urban areas. Urban resident's physical, psychological, social, and economic well-being fluctuate with the quality of their public transportation systems\cite{RichardJ01}.Currently, public transportation planning methods rely on human surveys to understand people's transportation needs. Despite the substantial time and cost spent on the survey process, the macroscopic analysis based on surveys is too static to reflect the fast development of urban areas\cite{Liu01}." Big data techniques can bridge this gap between static and dynamic planning methods and allow urban planners to keep pace with growing America urbanization rates. The United Nations estimates 87\% of Americans will live in urban environment by 2050 compared to 82\% in 2017\cite{Boyd01}. In a competitive market to attract the next generation of high-paying jobs and talented workforces, urban areas with optimized transportation systems can leverage their resident's higher QOL. 

Unfortunately, one major downfall of urbanization is frequent and intense traffic congestion due to more human activities within limited space, and consequently unnecessary energy consumption during traffic congestion{Liu01}. '"The Nobel Prize winning 2007 Intergovernmental Panel on Climate Change report concluded that greenhouse gas emissions must be reduced by 50\% to 85\% by 2050 in order to limit global warming to four degrees Fahrenheit\cite{Hodges01}.'' Optimizing public bus systems can play a major role in reducing greenhouse gas emissions. Compared to private vehicle transportation travel, average bus transit occupancy reduces CO2 emissions per passenger mile by 33.33\%, but when bus transit is fully occupied, CO2 emissions per passenger mile are reduced by 81.25\%\cite{Hodges01}. Therefore, to reduce greenhouse gas emissions and traffic congestion urban areas must attempt to convert users of private transportation to public bus transit. 

Although current mass transit bus systems offer various QOL, environmental, and safety benefits, the current public bus system is far from perfect. Fortunately, new technology makes it easier to collect live bus trip data such as velocity, position, heading, and number of riders and locale data such as traffic patterns, road networks, and points of interest. Applying this data to big data analysis techniques gives urban planners the knowledge to optimize their public transit systems. 



\section{Route \& Schedule Optimization Techniques}
\subsection{Human Mobility Mining \& Clustering}
\subsection{Ant Colony and Genetic Algorithms}
\subsection{Monte Carlo Simulations}

\section{Infrastructure Improvements}
\subsection{Digital Signage}
\subsection{Roundabouts}

\section{Conclusion}

Amid a major disruption in the transportation industry due to the introduction of self driving vehicles, public bus systems must adapt. For instance, public bus systems can leverage this new technology and create a network of smaller self driving buses that are highly optimized to local human mobility patterns. Using live traffic data, the Internet of Things, and fast computing big data algorithms, this system of public transportation could eliminate the need for private vehicle ownership all together. For several reasons, creating a highly optimized public bus system is good for QOL standards, sustainability, and safety. 

\begin{acks}

The author would like to thank Dr. Gregor von Laszewski for his support and suggestions to write this paper.

\end{acks}

\bibliographystyle{ACM-Reference-Format}
\bibliography{report} 

\end{document}
