\documentclass[sigconf]{acmart}

\usepackage{hyperref}

\usepackage{endfloat}
\renewcommand{\efloatseparator}{\mbox{}} % no new page between figures

\usepackage{booktabs} % For formal tables

\settopmatter{printacmref=false} % Removes citation information below abstract
\renewcommand\footnotetextcopyrightpermission[1]{} % removes footnote with conference information in first column
\pagestyle{plain} % removes running headers

\begin{document}
\title{Optimizing Mass Transit Bus Routes with Big Data}


\author{Matthew Schwartzer}
\affiliation{%
  \institution{Indiana University, School of Informatics, Computing, \& Engineering}
  \streetaddress{919 E 10th St}
  \city{Bloomington} 
  \state{Indiana} 
  \postcode{47408}
}
\email{mabschwa@indiana.edu}


\begin{abstract}
Big data analytics is important in the public bus transportation domain because optimized public bus systems can reduce congestion and greenhouse gas emissions, while offering a safe, affordable, and convenient way to travel; however, in many cities people prefer to take private transportation over public buses. Big data can help bridge this gap by optimizing the three pillars of public bus transportation: routes, schedules and infrastructure. 


\end{abstract}

\keywords{i523, hid225, \LaTeX, big data, bus route optimization}


\maketitle

\section{Introduction}

Optimized public bus transit, is one of the keys to bring sustainable development and a better quality of life to small and medium sized urban areas. Route optimization techniques like human mobility mining \& clustering, ant colony and genetic algorithms, and Monte Carlo simulations are only possible through the collection and analysis of big data. Currently "traditional public transportation planning methods have relied on human surveys to understand people's transportation needs. Despite the substantial time and cost spent on the survey process, the macroscopic analysis based on surveys is too static to reflect the fast development of urban areas\cite{Liu01}." 

Optimized public bus systems vastly improve residents quality of life (QOL) through four dimensions including improved physical, psychological, social, and economic well-being\cite{RichardJ01}. Consequently, poor transit experiences can have the exact opposite effect on residents QOL.  In a competitive environment to attract and retain residents, cities must prioritize a well optimized public bus system. In addition, a transportation systems needs flexibility to adapt to new points of interest due to American's continual urban migration. Approximately 82\% of Americans currently live in urban areas compared to just 64\% in 1950. This trend is expected to continue as the United Nations expects 87\% of Americans to live in urban environments by 2050\cite{Boyd01}. Optimized transportations systems are more important than ever due to millennial's tendencies to delay car ownership\cite{Etehad01}. Cities should work to prioritize non-car owners with mass bus transit options over less efficient, safe, and environmentally friendly private transportation alternatives.

One major downfall of urbanization is frequent and intense traffic congestion due to more human activities within limited space, and consequently unnecessary energy consumption during traffic congestion{Liu01}. The rise of global warming is putting pressure on urban planners to reduce greenhouse emissions. '"The Nobel Prize winning 2007 Intergovernmental Panel on Climate Change report concluded that greenhouse gas emissions must be reduced by 50\% to 85\% by 2050 in order to limit global warming to four degrees Fahrenheit\cite{Hodges01}.'' Compared to single occupancy vehicle (SOV) travel, average bus transit occupancy reduces CO2 emissions per passenger mile by 33.33\%. When bus transit is fully occupied, CO2 emissions per passenger mile are reduced by 81.25\% compared with SOV travel\cite{Hodges01}. 

Although current mass transit bus systems offer various QOL, environmental, and safety benefits, the current public bus system is far from perfect. Fortunately, new technology makes it easier to collect bus trip data such as velocity, position, heading, and number of riders and locale data such as traffic patterns, road networks, and points of interest. Applying this data to machine learning techniques gives urban planners the knowledge to optimize public transit systems with the end goal of converting private transportation users to public bus transit. 



\section{Route \& Schedule Optimization Techniques}
\subsection{Human Mobility Mining \& Clustering}
\subsection{Ant Colony and Genetic Algorithms}
\subsection{Monte Carlo Simulations}

\section{Infrastructure Improvements}
\subsection{Digital Signage}
\subsection{Roundabouts}

\section{Conclusion}

Amid a major disruption in the transportation industry due to the introduction of self driving vehicles, public bus systems must adapt. For instance, public bus systems can leverage this new technology and create a network of smaller self driving buses that are highly optimized to local human mobility patterns. Using live traffic data, the Internet of Things, and fast computing big data algorithms, this system of public transportation could eliminate the need for private vehicle ownership all together. For several reasons, creating a highly optimized public bus system is good for QOL standards, sustainability, and safety. 

\begin{acks}

The author would like to thank Dr. Gregor von Laszewski for his support and suggestions to write this paper.

\end{acks}

\bibliographystyle{ACM-Reference-Format}
\bibliography{report} 

\end{document}
